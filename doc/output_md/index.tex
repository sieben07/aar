

\documentclass[twopages]{scrartcl}

\usepackage[utf8x]{inputenc}

\usepackage[T1]{fontenc}
\usepackage{listings}
% \usepackage[ngerman]{babel}
\usepackage{longtable}
\usepackage{graphicx}

\usepackage{hyperref}

\usepackage{graphicx,grffile}
\makeatletter
\def\maxwidth{\ifdim\Gin@nat@width>\linewidth\linewidth\else\Gin@nat@width\fi}
\def\maxheight{\ifdim\Gin@nat@height>\textheight\textheight\else\Gin@nat@height\fi}
\makeatother
% Scale images if necessary, so that they will not overflow the page
% margins by default, and it is still possible to overwrite the defaults
% using explicit options in \includegraphics[width, height, ...]{}
\setkeys{Gin}{width=\maxwidth,height=\maxheight,keepaspectratio}




\hypersetup{colorlinks=true, linkcolor=black}

\begin{document}
\title{Activate all robots!}
\author{Orhan Küçükyılmaz}
\date{\today}
\maketitle
\newpage
\tableofcontents

\section*{Change History}

\begin{longtable}{|l|l|l|l|}
\hline
Version & Who & What & When\\
\hline
0.1 & Orhan Küçükyılmaz (OK) & Initial Document & 17.04.2014\\
0.2 & Orhan Küçükyılmaz (OK) & Updated Image & 08.07.2015\\
\hline
\end{longtable}

\section*{Abstract}

\begin{verbatim}a jump'n'shoot riddle game
\end{verbatim}

After the Hero's attack, it's your duty to\ldots{}

\begin{quote}
\emph{activate all robots!}
\end{quote}

Every \texttt{'activation-shoot'} costs you \texttt{points}.\newline
For every robot, machine or trap you activate,\newline
you get \texttt{points}.\newline
If you get hit, you lose a \texttt{point}.\newline
If you have only \emph{\texttt{one point left}}\newline
than you have only \texttt{one} \emph{ONE} shoot to make points or it's \ldots{}

\begin{quote}
\begin{quote}
\ldots{} \emph{GAME OVER!}
\end{quote}
\end{quote}

\begin{figure}
\begin{center}
\includegraphics{../output_md/assets/img/aar.png}
\end{center}
\caption{ His name is mini }

\end{figure}

\section{Levels, Robots, and more}\hypertarget{levels-robots-and-more}{}\label{levels-robots-and-more}

In this Section it's all about the levels the robots and more.

\subsection{Level 0 - The Start/Menu level}\hypertarget{level-0---the-startmenu-level}{}\label{level-0---the-startmenu-level}

Most games don't have a playable menu level. What is a playable menu level?, you ask.
Good question very good question indeed.

What is a menu?

Before a person can start a game he usually select from a menu what he wants
to do. The menu is usually something like:

\begin{itemize}
\item Start
\item Options
\item \ldots{}(something something)
\end{itemize}

Here the controls are different to the controls in the game.

Up and down on a joy-pad, joystick or on a keyboard (sometimes ``w'' for up
and ``s'' for down) toggle between the menu items. With one button on the
joy-pad, joystick or keyboard (sometime space or enter) the user select what
he wants to do.

In this game the menu is a playable level. Why?\newline
So the user uses the actual controls of the game and not some extra controls
for the menu. So he can learn the controls for the game early.

Level 0 teaches the player how to play the game, and presents him the first
robot to activate.

But first let us take a look at the elements of the first level:

\begin{figure}
\begin{center}
\includegraphics{../output_md/assets/img/title.png}
\end{center}
\caption{ The Title }

\end{figure}

\end{document}

